\section{4.1 導入}\label{4-1-}

一般的な非線形最適化問題は次の3つに分類される

\begin{enumerate}
\tightlist
\item
  一次元無制約問題
\item
  多次元無制約問題
\item
  多次元制約問題
\end{enumerate}

一つ目の問題は最も簡単に解け、三つ目の問題が一番難しい。
実際には多次元制約問題は通常多次元無制約問題に変形され、次に一次元無制約問題へと変形される。
事実上多くの利用可能な非線形計画アルゴリズムは無制約の1変数関数の最小化に基づいている。
それゆえ、効率的な多次元無制約最適化アルゴリズムと多次元制約最適化アルゴリズムを構築する際には、
効率的な一次元最適化アルゴリズムが要求されることになっている。

$ minimize F = f(x) $ ($f$は一変数関数)

この問題は$f(x)$がある範囲で単峰性関数である場合に解を持つ。
つまり、$ f(x) $がある範囲$ [x_L \leq x
\leq x_U] $ ($ x_L $と$ x_U $
はそれぞれ最小点$ x^* $の下限と上限) で$ f(x)
$が唯一の最小点を持つ場合、解を持つ。

一次元最適化法には一般的に二つの方法がある、即ち、探索法と近似法である。

探索法では$ x $ を含む $ [x_L , x_U] $
の範囲を関数の評価にしたがって縮小していき $ [ x_{L,k} ,
x_{U,k} ] $ なる充分に小さい範囲を得るまで縮小を繰り返す。
最小点は $ [ x_{L,k} , x_{U,k} ] $
の範囲の中間点に存在すると予想できる。
これらの方法はあらゆる関数に適用でき、$ f(x)
$の微分可能性は必要でない。
